\documentclass[10pt,a4paper,danish]{article}
%% Indlæs ofte brugte pakker
\usepackage{amssymb}
\usepackage[danish]{babel}
\usepackage[utf8]{inputenc}
\usepackage{listings}
\usepackage{fancyhdr}
\usepackage{hyperref}
\usepackage{booktabs}
\usepackage{graphicx}
\usepackage{todonotes}
\usepackage{algorithmic}
\usepackage{amsmath}


\pagestyle{fancy}
\fancyhead{}
\fancyfoot{}
\rhead{\today}
\rfoot{\thepage}

% Opsæt indlæsning af filer
\lstset{
 language=Python,
 extendedchars=\true,
 inputencoding=utf8,
 linewidth=\textwidth, basicstyle=\small,
 numbers=left, numberstyle=\footnotesize,
 tabsize=2, showstringspaces=false,
 breaklines=true, breakatwhitespace=false,
}

%% Titel og forfatter
\title{Aflevering 4 \\Algoritmer og datastrukturer\\Forår/Sommer 2011}
\author{Naja Mottelson (vsj465)\\Søren Pilgård (vpb984)}

%% Start dokumentet
\begin{document}

%% Vis titel
\maketitle
\newpage

%% Vis indholdsfortegnelse
\tableofcontents
\newpage

%% Rapport, baby!
\section{a}
\subsection{Algoritme}
Lader vi i og j være indicer til hhv. det første og sidste element i A, kan vi
skrive en multrådet algoritme til at lægge arrays sammen således: 
\\

\begin{algorithmic}[1]
\STATE SUM-ARRAYS$(A, B, C, i, j)$
\IF {$i == j$}
    \STATE $ C[i] = A[i] + C[i]$
\ELSE
    \STATE mid = $\lfloor (i + j) \rfloor$ 
    \STATE spawn SUM-ARRAYS$(A, B, C, i, mid)$
    \STATE SUM-ARRAYS$(A, B, C, mid + 1, j)$
	\STATE sync
\ENDIF
\end{algorithmic}

\subsection{Paralellisme}
Work for denne algoritme kan beskrives ved rekursionsligningen

$$
T_1(n) =
  \begin{cases}
    \Theta(1) & \text{hvis } i == j \\
   2T_1(\frac{n} {2}) + \Theta(1)& \text{hvis } i \neq j
  \end{cases}
$$
\\
Vi ser at det første led bidrager med $\Theta(\text{lg} n)$ arbejde, hvilket
dominerer det konstante led fra basistilældet. Altså har vi $T_1(n) =\Theta(\text{lg} n)$. 
Eftersom dybden af rekursive kald er logaritmisk, og de rekursive kald igen dominerer
den konstante faktor fra basistilfældet får vi at $T_\infty(n) = \Theta(\text{lg} n)$. 
Parallelismen for SUM-ARRAYS ($\frac{T_1} {T_\infty}$) bliver således $\frac{\text{lg n}} {\text{lg n}} = 1$. 

\section{c}
Eftersom hjælperoutinen ADD-SUBARRAY bidrager med $\Theta(n)$ som $grain-size$ vokser, kan en formel for SUM-ARRAYS' i termer af $n$ og $grain-size$ skrives på formen $$T_\infty = \lceil \frac{n} {grain-size} \rceil + n$$. 
\todo[inline] {Differentiér $T_\infty$, find nulpunkt og argumentér for at dette vil være den optimale værdi for grain-size.} 



\end{document} 

