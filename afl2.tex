\documentclass[10pt,a4paper,danish]{article}
%% Indlæs ofte brugte pakker
\usepackage{amssymb}
\usepackage[danish]{babel}
\usepackage[utf8]{inputenc}
\usepackage{listings}
\usepackage{fancyhdr}
\usepackage{hyperref}
\usepackage{booktabs}
\usepackage{graphicx}
\pagestyle{fancy}
\fancyhead{}
\fancyfoot{}
\rhead{\today}
\rfoot{\thepage}

% Opsæt indlæsning af filer
\lstset{
 language=Python,
 extendedchars=\true,
 inputencoding=utf8,
 linewidth=\textwidth, basicstyle=\small,
 numbers=left, numberstyle=\footnotesize,
 tabsize=2, showstringspaces=false,
 breaklines=true, breakatwhitespace=false,
}

%% Titel og forfatter
\title{Aflevering 1 \\Algoritmer og datastrukturer\\Forår/Sommer 2011}
\author{Naja Mottelson (vsj465)\\Søren Pilgård (vpb984)}

%% Start dokumentet
\begin{document}

%% Vis titel
\maketitle
\newpage

%% Vis indholdsfortegnelse
\tableofcontents
\newpage

%% HER STARTER RAPPORTEN
\section{a}
\begin{verbatim}
Hoare-partition(A,p=1,c=)
x = 13
i = -1
j = 13

Step 0:
i = -1, j = 13

     1  2  3  4  5  6  7  8  9 10 11 12
... ___________________________________ ...
:  |13|19| 9| 5|12| 8| 7| 4|11| 2| 6|21|  :
:..|__|__|__|__|__|__|__|__|__|__|__|__|..:
 |->i                              j<----|
i = 1, j = 11
SWAP(A[1], A[11])



Step 1:
i = 1, j = 11

     1  2  3  4  5  6  7  8  9 10 11 12
    ___________________________________ 
   | 6|19| 9| 5|12| 8| 7| 4|11| 2|13|21|
   |__|__|__|__|__|__|__|__|__|__|__|__|
    |->i                        j<-|

i = 2, j = 10
Swap(A[2], A[10])


Step 2:
i = 2, j = 10

     1  2  3  4  5  6  7  8  9 10 11 12
    ___________________________________ 
   | 6| 2| 9| 5|12| 8| 7| 4|11|19|13|21|
   |__|__|__|__|__|__|__|__|__|__|__|__|
       |                     j<-|
       |---------------------->i

i = 10, j = 9
return 9

\end{verbatim}
\section{b}
Vi ser at i og j initialiseres som positioner udenfor arrayret A[p..r] -hhv. som p-1 og r+1.
Grundet repeat-until konstruktionen vil de dog blive in/dekrementeret inden der laves opslag i A.

\subsection{Første iteration:}
\label{sec:foerste-it}
Første indre løkke (linje 5-7):
Vi ser at j dekrementeres ved indgangen til den første indre løkke. j ligger nu indenfor arrayet A[p..r]. 
Såfremt A[j] <= x stopper den indre løkke (linje 7), ellers fortsætter vi med at at dekrementere j.
vi ved at j aldrig bliver mindre end p, da A[p] = x. Den indre løkke vil altså altid stoppe med p >= j <= r samt A[j] <= x.
Derudover ved vi at der ikke kan være nogen elementer i A[j+1..r] der er mindre/lig med x.
 
Anden indre løkke (linje 8-10):
Ved indgangen til den anden indre løkke inkrementeres i, så i=p og da A[i] = A[p] = x vil den indre løkke stoppe her.

Efter de to indre løkker sammenlignes i og j.
Enten har j bevæget sig hele vejen ned igennem arrayet så j=i=p hvorefter vi ved at alle elementer er større end x. Der skal derfor ikke gøres mere da x er det mindste element og ligger forrest. Her vil funktionen terminere.

Hvis j istedet er forskellig fra i, må j nødvendigvis være større end i (j kan på nuværende tidspunkt ikke være mindre end i da A[i] = x).
Vi ved så at A[i] = x og at A[j] <= x. Derfor vil A[i] og A[j] blive ombyttet.
Efter første iteration gælder derfor at A[j] er et element der er mindre eller lig med x samt at A[i] = x. Derudover ved vi at alle elementer i A[j+1 .. r] er større end x.
Mere generelt kan vi sige at alle elementer i A[p..i] er mindre eller lig x samt at alle elementer i A[j .. r] er større eller lig med x.
Vi kan konkludere at vi på intet tidspunkt i første iteration tilgår et element der ikke findes i A[p..r].


\subsection{Efterfølgene iterationer:}
\label{sec:eft-it}
Vi ved nu at der i intervallet A[p..i] kun eksiterer elementer der er mindre/lig x samt at der i intervallet A[j..r] kun eksisterer elementer der er større/lig x
i hver iteration opretholder vi denne orden samtidig med at vi dekrementerer j og inkrementerer i.
%
Vi ved vi opretholder denne orden for i hvert trin i den første indre løkke lader vi j glide ned gennem arrayet mod p indtil den finder et element A[j] <= x der stopper den j, da den kun bevæger sig hen over elementer der er >= x ved vi at alle elementer i A[j+1..r] >= x
efterfølgende lader vi i gå mod r på samme måde som j. Hvis j møder et element der er >= x stoppes der. Vi ved derfor ligesom med j at elementerne i A[p..i-1] <= x. Når j og i er stoppet ved vi de skal ombyttes (med mindre  i>j hvormed vi skal stoppe funktionen som set under afslutningen). Efter ombytningen ved vi at A[j] >= x og at A[i] <= x. Vi har derfor at elementerne i A[p..i] <= x og A[j..r] >= x, samt at både j og i ligger mellem p og r da de stadig er på vej mod hinanden.
%
\subsection{Afslutning:}
\label{sec:afsl-it}
På et tidspunkt vil j og i bevæge sig forbi hinanden eller lande på samme element. 
Vi ved at A[p..i] efter første iteration indeholder mindst et element, samt at elementerne er mindre/lig x. Derfor vil den første indre løkke stoppe hvis j <= i da A[j] så vil være <= x

Tilsvarende ved vi at A[j..r] indeholder mindst et element samt at elementerne er større/lig x. Derfor vil den anden indre løkke stoppe hvis i >= j da A[i] så vil være >= x

Hvis bãde j og i stopper på det samme element, så j=i, må elementet være lig med x. Vi ved også algoritmen har traverseret hele arrayet A[p..(ij)..r]. Vi har så at alle elementerne A[p..i-1] <= x og at A[j+1..r] >=x samt at A[i]=A[j]=x.
Vi kan samtidigt konkludere at både i og j ikke har bevæget sig uden for arrayet da de har bevæget sig mod hinanden ind mod midten og stoppet på samme position.

Hvis ikke j og i stopper på samme plads vil de krydse hinanden.
Der kan nu ske en af to ting, j krydser i eller i krydser j

Det første scenarie vil forekomme når j rammer i, da vil j stoppe for vi har at A[i] <= x. Efter j er stoppet vil i rykke til j+1 hvor i også vil stoppe da A[i] så er >= x.
Det andet scenarie vil forekomme når j stopper på et tal der er <= x, herefter bevæger i sig så forbi j og lander på j+1 hvor det som før vil ende.
I begge scenarier har vi at invarianten ikke længere gælder for de nye værdier i og j da de nu har krydset hinanden. Vi ved dog stadig at alle elementer i arrayet A[p..i-1] <= x, vi ved også at i = j+1 derfor må der gælde at alle elementer A[p..j] <= x. Desuden ved vi at elementerne A[j+1..r] stadig må være >= x
Funktionen har derfor udført sin opgave og terminere. 

Udfra denne dybdegående gennemgang ser vi altså følgende:
\begin{itemize}
\item j og i starter udenfor intervallet p til r, men bliver de/inkrementeret som det første i de indre løkker så opslagene i A er gyldige.
\item I første trin rykkes j mod p indtil det finder et element der er mindre eller lig med x
\item Vi ved at dette altid vil lykkedes da x = A[p]
\item i inkrementeres med 1 og A[i] byttes ud med A[j]
\item Der ligger nu en værdi i hver ende af arrayet hvor j og i altid vil stoppe.
\item Hvis i og j krydser hinanden vil de på et tidspunkt møde en værdi der får dem til at stoppe og funktionen vil terminere da i >= j
\\

\item Det vistes at den ydre løkke afsluttes enten når i=j eller når i=j + 1\\
\textit{(I opgave c gennemgås hvordan j altid vil være skarpt mindre end r hvormed j+1 ikke vil falde uden for p til r)}
\end{itemize}


\section{c}
Den første indre løkke vil altid blive kørt mindst to gange: Ved første iteration bliver j sat til r. 
\begin{itemize}
\item Hvis j er større end x vil j fortsætte imod p, og j er derfor mindre end r.\item Hvis j er mindre eller lig med x, vil j blive ombyttet med i, og i næste iteration vil j blive skubbet så at j er mindre end r. 
\end{itemize}
 Således ved vi at j altid vil være mindre end r når funktionen terminerer. Dette ved vi fordi der herefter kan ske én af to ting: enten enten vil j møde et element der er mindre end eller lig med x hvorved der vil ske en ombytning. Ellers vil j møde et element der er større end x, i hvilket tilfælde j fortsætter mod p og er mindre end r. 

I de følgende iterationer bevæger j sig nedad imod p. Såfremt j møder et element der er mindre end eller lig med x vil j blive swappet, så j vil aldrig blive mindre end p, eftersom A[p] = x. j kan dog blive lig x i den situation hvor p er den største værdi i arrayet (jvf. argumentationen i sektion b). Ud fra dette kan vi konkludere at når funktionen terminerer vil den returnere en værdi som overholder ordenen p <= j < r.

\section{c}

p <= j < r
Vi ved at j altid vil være mindre end r når funktionen terminerer. Dette skyldes at den første indre løkke altid vil blive kørt mindst to gange: Ved første iteration bliver j sat til r. 




Der kan ske en af to ting, enten vil j være forskelig fra i hvorved der vil ske en ombytning




I første trin:
elementet mindre=, så skal der byttes, j != i
Der udføres en ekstra it. hvorved j < r

eller
j møder et element der er større så fortsætter j mod p og er <r




Dernæst skal vi vise
j <p
og at j kan ende i p (=)


\section{d}

Fra opgave b, sektion \ref{sec:foerste-it}, ved vi at efter første iteration vil det gælde at elementerne i A[p..i] er <= x samt at elementerne i A[j..r]
er >= x

vi ved også fra sektion \ref{sec:eft-it} at dette er gældene som j går mod p og i går mod r.

I sektion 


\section{e}
%\begin{algorithm}
  %\caption{Quicksort using Hoares partitioning algorithm}
  %\begin{algorithmic}
   % Quicsort(A, p, r)
    %\IF ($p < r$)
    %$j \gets $ Hoare-Partition(A, p, r)
  %\end{algorithmic}
%\end{algorithm}
\begin{verbatim}
QUICKSORT (A, p, r)
if p < r
       j = HOARE-PARTITION (A, p, r)
       QUICKSORT (A, p, j)
       QUICKSORT (A, j + 1, r)
\end{verbatim}
\end{document}