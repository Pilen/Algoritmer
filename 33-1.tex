\documentclass[10pt,a4paper,danish]{article}
%% Indlæs ofte brugte pakker
\usepackage{amssymb}
\usepackage[danish]{babel}
\usepackage[utf8]{inputenc}
\usepackage{listings}
\usepackage{fancyhdr}
\usepackage{hyperref}
\usepackage{booktabs}
\usepackage{graphicx}
%%Vi skal have ToDo-pakken fra Henne!

\pagestyle{fancy}
\fancyhead{}
\fancyfoot{}
\rhead{\today}
\rfoot{\thepage}

% Opsæt indlæsning af filer
\lstset{
 language=Python,
 extendedchars=\true,
 inputencoding=utf8,
 linewidth=\textwidth, basicstyle=\small,
 numbers=left, numberstyle=\footnotesize,
 tabsize=2, showstringspaces=false,
 breaklines=true, breakatwhitespace=false,
}

%% Titel og forfatter
\title{Aflevering 3 \\Algoritmer og datastrukturer\\Forår/Sommer 2011}
\author{Naja Mottelson (vsj465)\\Søren Pilgård (vpb984)}

%% Start dokumentet
\begin{document}

%% Vis titel
\maketitle
\newpage

%% Vis indholdsfortegnelse
%%\tableofcontents
%%\newpage

%% Rapport, baby!
\section{}
Til at finde lagene i de konvekse hylstre kan vi udnytte Jarvis' march-algoritmen.
Vi ved at alle punkter skal traverseres og bruges som knuder. Hvis et punkt ikke bruges
i et konvekst hylster vil det blive brugt i et af de følgende. I vores implementation
godtager vi at det inderste lag kan bestå af 1 eller 2 punkter. 

\paragraph{}
I vores algoritme bruger vi jarvis-march til at finde hvert lag i de konvekse hylstre. Vi 
indleder med at køre algoritmen på den samlede mængde af punkter. Dette gentages så på den
resterende mængde punkter, indtil alle punkterne er i ét lag.

\paragraph{}
Vi har så at hvert punkt i mængden $Q$ vil ligge som knude i et lag. Hvert punkt bruges kun 1 gang.

%%OBS! Tager Naja fejl her?
\paragraph{}
Vi ved at Jarvis' march-algoritmen kører i $\theta(n*h)$ hvor $n$ er antallet af punkter og $h$ er
antallet af knuder, og at $n$ bliver $h$ punkter mindre efter hver iteration. Samtidig ved vi at
alle punkter er knude i et lag, så  $h = n$, hvilket giver os at køretiden må være $\theta(n^2)$

\paragraph{}
Grundlæggende ved vi at hver jarvis-march iteration kigger på en knude og vinklerne til alle de 
resterende punkter. Da alle punkterne er knude i et lag, ser vi at jarvis-march skal bruges $h = n$ 
gange. Ergo må køretiden være $\theta(n*n)$. 

%% Meget fint, Søren. Du er sød og dygtig. 

%skriver det færdigt inden jeg går i seng/ inden jeg drikker mig fuld/ mens
\begin{verbatim}

convex-layers(Q)
    let S be an empty list

    while Q is not empty
        if lenght(Q) >= 3:
            k = jarvis-march(Q)
        else:
            k = Q //there is only 1 or 2 points left
        Q = Q-k
        S.append(k)

    return reverse(S)
\end{verbatim}
\end{document}
