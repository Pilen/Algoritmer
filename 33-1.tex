
\begin{document}

\begin{verbatim}
convex-layers(Q):
    let L be an empty stack of convex layers
    let p0 be lowest point in Q, leftmost in case of ties
    let p be some kind of list containing the remaining points in Q sorted by polar angle (dont remove points closer in case of points with same angle)
    let S be an empty stack

    while p is not empty:
        push(L, convex-layer())
convex-layer()
    let S be an empty stack
    push(p[0],S)
    push(p[
    return layer
\end{verbatim}

vi kan udnytte jarvis march \theta(n*h)
vi ved at alle punkter skal traverseres (da alle skal med i et lag) derfor er h=n
og hver knude leder alle n punkter igennem, derfor kan vi implementere den i \theta(n^2)

algoritmen er bare at køre jarvis march indtil alle punkterne er væk eller der er under 3 punkter tilbage

\begin{verbatim}
convex-layers(Q)
    let S be an empty list

    while Q is not empty
        if lenght(Q) >= 3:
            k = jarvis-march(Q)
        else:
            k = Q //there is only 1 or 2 points left
        Q = Q-k
        S.append(k)

    return reverse(S)



\end{verbatim}
\end{document}